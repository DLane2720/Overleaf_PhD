\documentclass[../main.tex]{subfiles}
\graphicspath{{\subfix{../images/}}}
\begin{document}

Real-time object detection lies at the core of autonomous perception, enabling vehicles to interpret their surroundings and respond safely to dynamic environments.
Effective detection requires identifying, classifying, and localizing objects rapidly enough to support decision-making and control loops, often under constraints of limited onboard computing and bandwidth.
In this work, these challenges are examined through vision and LiDAR detection pipelines implemented on the Minion sensing platform.

The central focus of this investigation is the development of a late-fusion framework that integrates information from both sensing domains to enhance detection robustness and confidence.
This framework represents the primary contribution of the research and demonstrates how visual and LiDAR detections can be combined at the decision level to exploit the complementary strengths of each sensing modality while mitigating their individual limitations.

To establish a meaningful basis for evaluation, two single-modality detection systems were selected as comparative baselines.
YOLOv8 represents the class of data-driven neural network detectors optimized for vision-based perception, while GB-CACHE represents the class of deterministic, geometry-based algorithms used for LiDAR clustering and segmentation.
These methods were not chosen to exhaust all available detection approaches, but because they are broadly representative of the state of the art within their respective sensing domains.

YOLOv8 was selected for its status as one of the most widely adopted and optimized real-time vision networks, capable of achieving high detection accuracy and efficient inference on GPU-based platforms.
Its inclusion provides a mature and well-understood benchmark for camera-based perception and a reliable reference for evaluating the complementary benefits of late fusion.

GB-CACHE was selected as a deterministic counterpart to YOLO that performs real-time clustering and segmentation without the computational overhead or data dependency of neural network–based LiDAR methods.
Its use of concave-hull extraction provides boundary representations more meaningful for maritime navigation than the simplified 3D bounding boxes produced by models such as PointNet or VoxelNet.
This geometry-driven approach broadens the scope of the comparative study, especially given that the advantages of neural LiDAR methods for real-time deployment remain unproven.

These two approaches represent two distinct strategies for real-time perception, and the late-fusion method developed in this work integrates their complementary strengths to achieve higher confidence detections across a range of operating conditions.
The following sections examine each detection method in sequence.
Section~\ref{yolo} presents the YOLOv8 vision-based detector and its implementation for real-time object recognition.
Section~\ref{gbcache} describes the GB-CACHE algorithm for LiDAR-based clustering and object segmentation.
Finally, Section~\ref{late_fusion} introduces the late-fusion framework developed in this research and evaluates its performance relative to the single-modality detectors through a comparative analysis of detection accuracy, computational efficiency, and real-time operation.

\end{document}