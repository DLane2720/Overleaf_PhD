\documentclass[../main.tex]{subfiles}
\graphicspath{{\subfix{../images/}}}
\begin{document}

The rise of autonomous systems is transforming various industries, and the maritime sector is no exception. Autonomous surface vessels (ASVs) have the potential to improve safety in traditionally human-operated vessels, taking on dull, dirty, and dangerous jobs.
A critical component of these systems is the ability to accurately perceive the surrounding environment. 
Real-time object detection and classification are essential for autonomous navigation and situational awareness, ensuring safe and efficient operation in maritime environments. 
While the automotive industry has seen significant advances in autonomous vehicle technology, these have yet to be fully realized for marine ASVs.
This translates directly into the volume of literature published on automotive and marine object detection strategies. 
\textcolor{red}{While detection strategies can be adapted from the automotive environment, the marine environment is less structured yet retains significant complexity and object density, particularly in littoral zones.}
Maritime object detection is challenging due to turbulent waves, sea fog, water reflection, and fluctuating lighting conditions. 
For close to mid-range detection, typical marine radar sensors provide poor fidelity.
Cameras and LiDAR provide much denser information but can still struggle in inclement weather or poor lighting conditions. 
These limitations highlight the need for more robust systems and highlight where combining the data from multiple sensors through data fusion can provide significant performance benefits. 
However, more research is required to find optimal real-time fusion strategies for maritime applications.

This study investigates the comparative performance of LIDAR and camera-based object detection and classification systems in well-lit maritime environments with near-optimal weather conditions. 
It examines sensor fusion strategies, training efficiency, and performance metrics on a range of common maritime objects, including navigational markers and small to medium-sized vessels, and aims to find an efficient real-time fusion strategy that leverages the best performance of each sensor.

% The novel contributions of this paper may be broken down as:
% \begin{itemize}
%     \item 
% \end{itemize}

% \section{Introduction} \label{introduction}

\end{document}