\documentclass[../main.tex]{subfiles}
\graphicspath{{\subfix{../images/}}}
\begin{document}

%%%%%%%%    %%%%%%%%    %%%%%%%%    %%%%%%%%
% Note: All in-text citations should appear as \cite{einstein}

\Acp{USV} have emerged as essential platforms capable of performing dangerous, dirty, and cumbersome tasks that exceed human capability. These vessels are pivotal in various maritime operations, including environmental monitoring, search and rescue missions, and resource exploration \cite{liebergall, eckstein2024}.% [1], [2]. 
Their ability to operate independently with minimal human intervention has significantly enhanced operational efficiency and safety at sea \cite{bai2022}.%[3].

Autonomous vehicles use a variety of sensors to perceive their surroundings, but primarily rely on some combination of visual information through a camera and spatial data provided by \ac{LiDAR} \cite{yeong2021}.%[4].
Each sensing modality offers distinct advantages: visual data provides rich color and texture information, while \ac{LiDAR} delivers precise spatial measurements of the surrounding environment.
Real-time object detection methods have been developed for both sensing modalities, leveraging deep learning architectures.
Object detection with visual data often employs transfer learning on pre-trained convolutional neural networks such as ResNet \cite{he2016} and \ac{YOLO} \cite{ultralytics}.%[6].
Similarly, \ac{LiDAR}-based object detection can be performed using point-based algorithms like PointNet \cite{garcia-garcia2016}, voxel-based methods such as VoxelNet \cite{zhou2018}, or hybrid approaches like PV-RCNN \cite{shi2021}.%[9].
Despite these advancements, each modality has inherent limitations—vision-based systems struggle with poor lighting conditions and occlusions, while \ac{LiDAR} data can be sparse and affected by water reflections.

To address these limitations, sensor fusion techniques have been explored as a means of combining the strengths of both modalities. 
Research into sensor fusion methods dates back to military applications in the 1980s and has gained significant traction in the last 15 years, particularly due to interest from the automotive industry in autonomous driving technologies. However, no unified approach has been established for optimal sensor fusion, with ongoing debates regarding the best fusion strategies (e.g., early, mid, or late fusion) and their trade-offs concerning computational efficiency and accuracy.

While research in the automotive sector has contributed significantly to sensor fusion methodologies \cite{yeong2021,clunie2021,roriz2022,cui2022,das2022,liu2023a}, direct application to maritime environments remains challenging due to fundamental environmental differences. 
Automotive environments are highly structured, with well-defined lanes, uniform object classes, and relatively predictable lighting conditions. 
In contrast, the maritime environment introduces additional complexities, including dynamic vehicle motion induced by wind and waves, variable scene density (ranging from sparse open waters to congested littoral zones), and specular reflections on the water surface that can interfere with both vision-based \cite{liu2023a} and \ac{LiDAR}-based object detection \cite{ahmed2024}.%[15]. 
These factors necessitate domain-specific adaptations of sensor fusion architectures to ensure robust real-time object detection for \acp{USV}. 
However, the lack of available maritime-specific datasets \cite{kim2022,su2023,thompson2023} creates an additional challenge.

Given these challenges, further research is needed to enhance sensor fusion methodologies for maritime applications. 
Key areas of investigation include efficient feature selection tailored to maritime object classes, the development of lightweight fusion architectures suited for real-time processing, and an evaluation of computational requirements for deployment on \ac{USV} hardware. 
Addressing these research gaps will contribute to the advancement of autonomous maritime perception, enhancing the operational capabilities of \acp{USV} in complex and dynamic environments.

\end{document}