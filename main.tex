\documentclass{erauthesis}
\department{Mechanical Engineering}
\chair{Eric Coyle, Ph.D}
\dean{James Gregory, Ph.D.}
\dgc{Lon Moeller, J.D.}
\depchair{Patrick Currier, Ph.D.}
\advisortitle{Committee chair}
\usepackage{graphicx}
\usepackage{amsmath}
\usepackage{array}
\usepackage{enumitem}
% \usepackage[style=authoryear]{biblatex} % or numeric, apa, etc.
% \addbibresource{Dissertation.bib}         % your Zotero export

% acronyms
\usepackage{acronym} 

\title{A STUDY IN OBJECT DETECTION AND CLASSIFICATION
PERFORMANCE BY SENSING MODALITY FOR AUTONOMOUS
SURFACE VESSELS} % the title should be included here
\author{Daniel P. Lane} 
\graduation{December}{2025}
\advisor {Eric Coyle} %Committe chair


\coadvisor{Subhradeep Roy} % If you do not have a co-advisor, delete this whole command

\committememzero{Xxxx X. Xxxxxxxxx, Ph.D.} % If you have a co-advisor, do not edit this member name
%% Enter the name of the committee members
\committememone{Patrick Currier}
\committememtwo{Monica Garcia}
\committememthree{Jianhua Liu}
\committememfour{TBD}



%\signaturepush{-2.0}									

\begin{document}

\frontmatter

\maketitle

% \makesignature
\makeatletter 
\advance\fau@frontstage by 1  % Skip signature page but maintain counter
% \makeanother
\begin{acronym}[GB-CACHE] % Give the longest label here so that the list is nicely aligned
\acro{AGS}{autonomous ground system}
\acro{ASV}{autonomous surface vessel}
\acro{USV}{unmanned surface vessel}
\acro{ERAU}{Embry-Riddle Aeronautical University}
\acro{GB-CACHE}{grid-based clustering and concave hull extraction}
\acro{GPS}{Global Positioning System}
\acro{WAM-V}{wave-adaptive modular vessel}
\acro{SDR}{standard dynamic range}
\acro{HDR}{high dynamic range}
\acro{DOL-HDR}{digital-overlap HDR}
\acro{IMU}{inertial measurement unit}
\acro{INS}{inertial navigation system}
\acro{IoU}{intersection over union}
\acro{LiDAR}{light detection and ranging}
\acro{pps}{points per second}
\acro{mAP}{mean average precision}
\acro{MPC}{model predictive control}
\acro{RGB}{red, green, blue}
\acro{ROI}{region of interest}
\acro{ROS}{Robot Operating System}
\acro{YOLO}{You Only Look Once}
\acro{YOLOv8}{You Only Look Once ver. 8.0}
\acro{LWIR}{long-wave infrared}
\acro{FPS}{frames per second}
\acro{EFL}{effective focal length}
\acro{FOV}{field of view}
\acro{LAN}{local area network}
\acro{SEI}{supplemental enhancement information}
\acro{NAL}{network abstraction layer}
\acro{NTP}{Network Time Protocol}
\acro{PTP}{Precision Time Protocol}
\acro{RTP}{Real-time Transport Protocol}
\acro{RTSP}{Real-time Streaming Protocol}
\acro{UDP}{User Datagram Protocol}

\end{acronym}

\begin{acknowledgements}

	% \raggedright XXxxxx xxxxx xxxxx xxxxx xxxxx xxxxx xxxxx.  Xxxxx xxxxx xxxxx xxxxx xxxxx xxxxx xxxxx xxxxx xxxxx xxxxx xxxxx xxxxx xxxxx xxxxx xxxxx xxxxx xxxxx xxxxx.\\Xxxxx xxxxx xxxxx xxxxx xxxxx xxxxx xxxxx xxxxx xxxxx xxxxx xxxxx xxxxx xxxxx xxxxx xxxxx xxxxx xxxxx xxxxx xxxxx xxxxx.  Xxxxx xxxxx xxxxx xxxxx xxxxx xxxxx xxxxx xxxxx xxxxx xxxxx xxxxx xxxxx xxxxx xxxxx xxxxx.\\
    \raggedright In addition to any personal statements of acknowledgement, be sure to include any acknowledgement statements required by research sponsors.\\{[Single page limit]} 

    \raggedright This research was sponsored in part by the Department of the Navy, Office of Naval Research through ONR N00014-17-1-2492, and the Naval Engineering Education Consortium (NEEC) through grants N00174-19-1-0018 and N00174-22-1-0012, sponsored by NSWC Carderock and NUWC Keyport respectively. Any opinions, findings, conclusions, or recommendations expressed in this material are those of the authors and do not necessarily reflect the views of the Department of the Navy or Office of Naval Research.
\end{acknowledgements}

\begin{abstract}
	\raggedright Researcher: Daniel P. Lane
 \\Title: A study in object detection and classification performance by sensing modality for autonomous surface vessels \\Institution:	Embry-Riddle Aeronautical University\\Degree:	Doctor of Philosophy in Mechanical Engineering\\Year:	2025 \\
 This research addresses the critical gap in quantitative performance comparison between \ac{LiDAR} and vision-based sensing for real-time maritime object detection on autonomous surface vessels.
 Using \ac{ERAU}'s Minion platform and 2024 Maritime RobotX Challenge data, this study evaluates \ac{GB-CACHE} \ac{LiDAR} processing against \ac{YOLO} vision detection across six maritime object categories. The methodology encompasses real-time performance analysis, multi-sensor calibration, and sensor fusion for bounding box confidence integration.
 Performance metrics include precision, recall, \ac{mAP}, training requirements, and computational efficiency. 
 % Results demonstrate [key performance finding] and establish [fusion outcome]. 
 The research provides quantitative baselines for maritime sensing modality selection and validated calibration procedures enabling improved autonomous navigation in complex maritime environments.
 
 % Lorem ipsum dolor sit amet... This is a summative abstract, not just a list of topics.  Include relevant information including conclusions and recommendations.  Limit to 150 words; spell out abbreviations; citations not needed.

\end{abstract}
\pagetableofcontents
\clearpage
\listoftables					% Or \nolistoftables if there are no 
\clearpage
\listoffigures					% Or \nolistoffigures if there are no 



\mainmatter
\newpage
\chapter{Introduction}

\section{Introduction}

\subsection{Significance of Study}

The development of \acp{ASV} represents a significant technological advancement requiring sophisticated perception systems capable of detecting and classifying maritime objects with high accuracy and reliability in real-time operational environments. As \ac{ASV} technology advances toward practical deployment in commercial and research applications, understanding the comparative performance characteristics of different sensing modalities and their integration through sensor fusion methodologies becomes critical for effective system design and operational safety assurance.

The advancement of autonomous surface vessel technology has gained significant momentum through comprehensive research programs and competitive evaluation platforms such as the Maritime RobotX Challenge, where multidisciplinary teams develop and test \ac{ASV} platforms under realistic maritime operational conditions. These research platforms serve as essential testbeds for evaluating perception technologies under actual environmental constraints, providing valuable empirical insights into sensor performance characteristics, real-time processing capabilities, and complex system integration challenges that cannot be adequately assessed through simulation alone.

Research platforms provide critical opportunities for real-world validation of perception algorithms under actual maritime conditions with operational time constraints. These platforms enable comprehensive comparative analysis opportunities for evaluating different sensor modalities and advanced sensor fusion approaches under controlled yet realistic testing scenarios. Furthermore, research platforms facilitate essential technology transfer pathways from experimental research systems to operational \ac{ASV} deployments requiring robust real-time performance guarantees. Finally, these platforms enable systematic performance benchmarking that supports rigorous evaluation of detection accuracy, classification reliability, and sensor fusion effectiveness across diverse maritime operational scenarios.

Current \ac{ASV} development faces a significant knowledge gap regarding the quantitative performance characteristics of different sensing modalities and their integration methodologies in complex maritime environments. While individual sensor technologies, particularly \ac{LiDAR} systems and vision-based cameras, have been extensively studied and validated in terrestrial applications, their comparative performance characteristics and sensor fusion integration capabilities in maritime contexts lack systematic quantitative analysis with emphasis on real-time processing constraints and computational efficiency requirements.

Comprehensive performance analysis requires systematic detection accuracy comparison between \ac{LiDAR}-based systems utilizing \ac{GB-CACHE} processing and vision-based systems implementing \ac{YOLO} object detection algorithms. Additionally, rigorous classification performance evaluation across diverse maritime object types using advanced machine learning algorithms must be conducted to establish performance baselines. Assessment of training requirements for machine learning-based classification systems, particularly focusing on data efficiency and convergence characteristics, represents another critical analytical requirement. Real-time processing capabilities and computational efficiency evaluation under operational constraints must be systematically analyzed to ensure practical deployment feasibility. Moreover, sensor fusion effectiveness evaluation, specifically examining bounding box confidence integration methodologies, requires comprehensive analysis to determine optimal multi-modal processing approaches. Finally, environmental robustness evaluation across varying maritime conditions, including different weather states, lighting conditions, and sea states, must be conducted to ensure reliable operational performance.

\ac{ASV} perception systems must reliably detect and classify a diverse range of maritime objects that are critical for safe autonomous navigation, requiring robust algorithms capable of real-time processing under challenging environmental conditions. The maritime environment presents unique detection challenges due to varying lighting conditions, wave-induced platform motion, and the diverse physical characteristics of navigation-critical objects that must be accurately identified and classified. Navigation buoys, including Polyform A-2 and A-3 buoys available in various colors for channel marking and navigation guidance, represent primary detection targets requiring high accuracy classification. Regulatory markers, specifically Sur-Mark tall buoys designed for navigation reference and hazard identification, present distinct detection challenges due to their geometric characteristics and operational deployment contexts. Light towers serving as active navigation aids provide both visual and electronic guidance signals, requiring detection algorithms capable of handling variable illumination and signaling states. Various vessels, including recreational boats and commercial watercraft, represent dynamic detection targets with diverse size, shape, and motion characteristics that complicate reliable classification. Maritime infrastructure elements, including docks, piers, and other fixed navigational hazards, require robust detection capabilities to ensure safe autonomous navigation in complex harbor and coastal environments.

% \subsection{Problem Statement}
\subsection{Problem Statement: Performance Comparison Gap}

Despite the growing operational importance of autonomous surface vessels and the significant maturation of individual sensor technologies over the past decade, there exists a critical and well-documented gap in quantitative performance comparison between different sensing modalities specifically applied to maritime object detection and classification tasks. Current \ac{ASV} development efforts lack systematic analytical frameworks for evaluating how \ac{LiDAR}-based systems utilizing advanced point cloud processing algorithms perform relative to vision-based systems implementing state-of-the-art deep learning approaches when deployed in realistic maritime operational environments.

Existing research efforts in maritime object detection and classification have primarily focused on individual sensor implementations and algorithm development without conducting comprehensive comparative analysis that would inform optimal sensor selection and integration strategies for operational \ac{ASV} systems. Contemporary research demonstrates a predominant focus on \ac{LiDAR}-only implementations that emphasize point cloud processing methodologies and clustering algorithms specifically adapted for maritime environments, yet these studies typically lack comparative evaluation against alternative sensing modalities. Similarly, vision-only system research emphasizes deep learning approaches for maritime object recognition, particularly convolutional neural network architectures, but generally operates in isolation without systematic comparison to \ac{LiDAR}-based approaches. The limited cross-modal comparison studies that do exist provide insufficient quantitative performance metrics and lack standardized evaluation frameworks necessary for meaningful comparative analysis. Furthermore, there exists a notable absence of standardized evaluation frameworks specifically designed for maritime perception systems, hindering systematic comparison and technology advancement across research groups and commercial developers.

The absence of comprehensive quantitative performance analysis leaves fundamental technical and operational questions unanswered for \ac{ASV} system designers and engineers responsible for developing reliable autonomous navigation systems. Critical questions regarding detection and classification performance remain inadequately addressed in current research literature. Specifically, the comparative detection accuracy performance of \ac{LiDAR}-based systems utilizing \ac{GB-CACHE} processing versus vision-based systems implementing \ac{YOLO} algorithms for specific maritime object types requires systematic investigation. The precision and recall characteristics of each sensing modality across different object classes under varying environmental conditions need quantitative evaluation to inform sensor selection decisions. Training requirements, including data volume, computational resources, and convergence time, differ significantly between \ac{LiDAR} feature-based approaches and vision-based deep learning methodologies, yet these differences lack systematic quantification. Computational overhead associated with each sensing modality for real-time operation, including processing latency and resource utilization, requires comprehensive analysis to ensure practical deployment feasibility. Additionally, the effectiveness of sensor fusion methodologies, particularly bounding box confidence integration approaches, needs rigorous evaluation to determine optimal multi-modal processing strategies for enhanced detection performance.

% \subsection{Problem Statement}
\subsection{Problem Statement: Sensor Fusion Challenges}

Autonomous surface vessels require precise integration of multiple sensing modalities to achieve reliable object detection and classification performance that meets operational safety standards for autonomous navigation. A fundamental and persistent challenge in \ac{ASV} perception system development lies in establishing and maintaining accurate spatial and temporal calibration between \ac{LiDAR} and camera systems under dynamic maritime operational conditions that present unique environmental challenges not encountered in terrestrial applications.

Multi-sensor \ac{ASV} platforms face unique spatial calibration requirements that differ significantly from terrestrial applications due to the dynamic nature of maritime environments and the continuous mechanical stresses imposed by marine operational conditions. Environmental factors affecting calibration present ongoing challenges for maintaining sensor alignment accuracy. Platform motion induced by wave action creates continuous roll, pitch, and yaw movements that affect sensor alignment and require robust calibration maintenance strategies. Vibration and mechanical stress inherent in marine environments cause gradual calibration drift that can degrade sensor fusion performance over extended operational periods. Temperature variations in maritime environments affect sensor mounting structures and optical characteristics, potentially introducing systematic calibration errors that must be compensated through adaptive calibration procedures. Saltwater corrosion presents long-term challenges by potentially altering sensor mounting hardware characteristics over extended deployment periods, requiring regular calibration validation and maintenance protocols.

Precision requirements for effective sensor fusion establish demanding performance specifications for calibration maintenance systems that must operate reliably under dynamic maritime conditions. Sub-pixel accuracy calibration is essential for accurate \ac{LiDAR} point cloud projection onto camera images, enabling effective correlation between sensor modalities for object detection applications and ensuring that spatial relationships between sensors remain consistent across operational scenarios. Millimeter-level precision in spatial calibration is required for effective object detection correlation between \ac{LiDAR} and vision systems, particularly for small maritime objects such as navigation buoys where detection accuracy directly impacts navigation safety. Consistent calibration maintenance across varying operational conditions, including different sea states and weather conditions, requires robust calibration validation procedures that can adapt to changing environmental parameters. Real-time calibration validation capabilities are necessary for detecting calibration degradation during operation and implementing corrective measures to maintain sensor fusion performance without interrupting autonomous navigation operations.

\subsection{Limitations and Assumptions}

This research investigation is conducted using \ac{ERAU}'s Minion \ac{ASV} research platform and utilizes sensor data collected during the 2024 Maritime RobotX Challenge competition, which establishes specific operational and methodological constraints that define the scope and applicability of research findings.

Platform-specific limitations inherent in this research approach must be acknowledged and considered when interpreting results and their broader applicability. Research findings are specifically applicable to \ac{ERAU}'s Minion research platform design, including its particular sensor mounting configuration, platform dynamics characteristics, and operational capabilities, which may not directly translate to other \ac{ASV} platform designs. The competition environment context, where primary data collection occurred during RobotX challenge events, may not represent the full spectrum of maritime operational conditions encountered in commercial or military \ac{ASV} deployments. Geographic constraints imposed by conducting testing in competition and training areas introduce environmental characteristics that may not be representative of other maritime operational regions. Operational scenarios focused on RobotX challenge tasks may not encompass the complete range of potential \ac{ASV} mission requirements and environmental conditions encountered in practical autonomous vessel operations.

This research focuses on specific maritime object categories that are relevant to RobotX competition scenarios and representative of typical \ac{ASV} navigation challenges, while acknowledging that maritime environments contain additional object types not addressed in this investigation. The research addresses six primary object classes that represent critical navigation elements in maritime environments. Tall buoys, specifically Sur-Mark regulatory buoys with standardized dimensions of 39-inch height, 10-inch column diameter, and 18-inch ring diameter, represent regulatory navigation markers requiring reliable detection and classification. Polyform A-2 buoys, measuring 14.5 inches by 19.5 inches and available in red, green, blue, and black color variants, serve as channel markers and navigation references requiring color-specific classification capabilities. Polyform A-3 buoys, with dimensions of 17 inches by 23 inches and identical color availability, represent larger navigation buoys requiring robust detection across varying distances and environmental conditions. Light towers function as active navigation aids incorporating electronic and visual signaling capabilities, presenting detection challenges due to variable illumination states and complex geometric structures. Jon boats, characterized as flat-bottom chase boats utilized in competition and training scenarios, represent small vessel detection targets with distinct geometric and motion characteristics. Sailboats, including recreational sailing vessels commonly encountered in competition environments, represent larger vessel detection targets with variable configuration due to sail and rigging arrangements.

\textbf{Definitions of Terms}

\begin{itemize}[label={}]
    \item\textbf{Autonomous Surface Vessel (ASV)} An unmanned watercraft capable of independent navigation and task execution without direct human control, utilizing onboard sensors and computational systems for environmental perception and decision-making.
    
    \item\textbf{Clustering} A computational technique that groups data points with similar characteristics or spatial proximity to identify distinct objects or regions within complex datasets.

    \item\textbf{Grid-Based Clustering} A spatial data organization methodology that partitions three-dimensional point cloud data into regular grid structures to facilitate efficient clustering analysis and object identification within defined spatial regions.
    
    \item\textbf{Concave Hull} A geometric boundary that closely follows the shape of a point set by allowing inward curves, providing a more accurate representation of object boundaries compared to convex hull approaches.
    
    \item\textbf{You Only Look Once (YOLO)} A real-time object detection algorithm that processes entire images in a single forward pass through a convolutional neural network, simultaneously predicting bounding boxes and class probabilities for detected objects.
    
    \item\textbf{Sensor Fusion} The computational process of combining data from multiple sensors to produce more accurate, reliable, or comprehensive information than could be achieved using individual sensors independently.
    
    \item\textbf{Bounding Box} A rectangular region that defines the spatial boundaries of a detected object within an image or three-dimensional space, typically specified by corner coordinates or center point with width and height dimensions.
    
    \item\textbf{Confidence Integration} A methodology for combining detection results from multiple sensors by evaluating and integrating the confidence scores associated with object predictions to improve overall detection reliability.
    
    \item\textbf{Maritime RobotX Challenge} An international autonomous surface vessel competition that provides standardized testing scenarios and performance evaluation frameworks for ASV perception, navigation, and manipulation capabilities.
    
    \item\textbf{Real-time Processing} Computational processing that guarantees response within specified time constraints, typically requiring completion of detection and classification tasks within predetermined latency limits suitable for autonomous navigation safety requirements.
    
    \item\textbf{Light Detection and Ranging (LiDAR)} A remote sensing technology that uses laser pulses to measure distances and create detailed three-dimensional point cloud representations of environmental features and objects.
    
    \item\textbf{Point Cloud} A collection of data points in three-dimensional space representing the external surface of objects, typically generated by LiDAR sensors through distance measurements to environmental features.
\end{itemize}

\textbf{List of Acronyms}

\begin{itemize}[label={}]
    \item\textbf{ASV} Autonomous Surface Vessel
    \item\textbf{ERAU} Embry-Riddle Aeronautical University  
    \item\textbf{GB-CACHE} Grid-Based Clustering and Concave Hull Extraction
    \item\textbf{LiDAR} Light Detection and Ranging
    \item\textbf{RGB} Red, Green, Blue
    \item\textbf{ROS} Robot Operating System
    \item\textbf{YOLO} You Only Look Once
    \item\textbf{IoU} Intersection over Union
    \item\textbf{mAP} mean Average Precision
    \item\textbf{ROI} Region of Interest
    \item\textbf{GPS} Global Positioning System
    \item\textbf{IMU} Inertial Measurement Unit
\end{itemize}

\chapter{Review of the Relevant Literature}

% Note: All in-text citations should appear as \cite{einstein}

\acp{USV} have emerged as essential platforms capable of performing dangerous, dirty, and cumbersome tasks that exceed human capability. These vessels are pivotal in various maritime operations, including environmental monitoring, search and rescue missions, and resource exploration \cite{liebergall, eckstein2024}.% [1], [2]. 
Their ability to operate independently with minimal human intervention has significantly enhanced operational efficiency and safety at sea \cite{bai2022}.%[3].

Autonomous vehicles use a variety of sensors to perceive their surroundings but primarily rely on some combination of visual information through a camera and spatial data provided by \ac{LiDAR} \cite{yeong2021}.%[4].
Each sensing modality offers distinct advantages: visual data provides rich color and texture information, while \ac{LiDAR} delivers precise spatial measurements of the surrounding environment.
Real-time object detection methods have been developed for both sensing modalities, leveraging deep learning architectures.
Object detection with visual data often employs transfer learning on pre-trained convolutional neural networks such as ResNet \cite{he2016} and \ac{YOLO} \cite{ultralytics}.%[6].
Similarly, \ac{LiDAR}-based object detection can be performed using point-based algorithms like PointNet \cite{garcia-garcia2016}, voxel-based methods such as VoxelNet \cite{zhou2018a}, or hybrid approaches like PV-RCNN \cite{shi2021}.%[9].
Despite these advancements, each modality has inherent limitations—vision-based systems struggle with poor lighting conditions and occlusions, while \ac{LiDAR} data can be sparse and affected by water reflections.

To address these limitations, sensor fusion techniques have been explored as a means of combining the strengths of both modalities. 
Research into sensor fusion methods dates back to military applications in the 1980s and has gained significant traction in the last 15 years, particularly due to interest from the automotive industry in autonomous driving technologies. However, no unified approach has been established for optimal sensor fusion, with ongoing debates regarding the best fusion strategies (e.g., early, mid, or late fusion) and their trade-offs concerning computational efficiency and accuracy.

While research in the automotive sector has contributed significantly to sensor fusion methodologies \cite{yeong2021,clunie2021,roriz2022,cui2022,das2022,liu2023a}, direct application to maritime environments remains challenging due to fundamental environmental differences. 
Automotive environments are highly structured, with well-defined lanes, uniform object classes, and relatively predictable lighting conditions. 
In contrast, the maritime environment introduces additional complexities, including dynamic vehicle motion induced by wind and waves, variable scene density (ranging from sparse open waters to congested littoral zones), and specular reflections on the water surface that can interfere with both vision-based \cite{liu2023a} and \ac{LiDAR}-based object detection \cite{ahmed2024}.%[15]. 
These factors necessitate domain-specific adaptations of sensor fusion architectures to ensure robust real-time object detection for \acp{USV}. 
However, the lack of available maritime-specific datasets \cite{jun-hwa2022,su2023,thompson2023} creates an additional challenge.

Given these challenges, further research is needed to enhance sensor fusion methodologies for maritime applications. 
Key areas of investigation include efficient feature selection tailored to maritime object classes, the development of lightweight fusion architectures suited for real-time processing, and an evaluation of computational requirements for deployment on \ac{USV} hardware. 
Addressing these research gaps will contribute to the advancement of autonomous maritime perception, enhancing the operational capabilities of \acp{USV} in complex and dynamic environments.

\chapter{Sensing Platform}

    \section{USV Platform}

        \subsection{Sensors}
        
            \subsubsection{Perception Geometry}

The sensor configuration on the Minion \ac{ASV} was designed to maximize spatial and temporal overlap between vision and \ac{LiDAR} sensing modalities, enabling robust multi-modal object detection and sensor fusion.
The perception geometry emphasizes forward-facing coverage optimized for maritime navigation scenarios, where objects of interest primarily appear ahead of the vessel during transit operations.

The multi-modal perception system integrates complementary sensor types to leverage the strengths of each modality while compensating for individual limitations.
High dynamic range cameras provide dense pixel-level information with rich texture and color features, while \ac{LiDAR} sensors deliver precise three-dimensional spatial measurements independent of lighting conditions.
The geometric arrangement ensures that both sensor types observe the same spatial volume, a prerequisite for effective late fusion methodologies.

The perception system employs three Leopard Imaging IMX490 \ac{HDR} cameras arranged to provide overlapping horizontal coverage.
Each camera features a 65-degree horizontal field of view, positioned with strategic overlap to create a composite forward-facing perception envelope.
This configuration yields comprehensive visual coverage of the navigation corridor ahead of the vessel while maintaining sufficient overlap for stereoscopic depth estimation if required in future work.
Complementing the camera array, three Livox Horizon \ac{LiDAR} units provide spatial measurement coverage designed to maximize overlap with the camera field of view.
Each Livox Horizon features an 81.7° × 25.1° (horizontal × vertical) field of view.
The three units are positioned to deliver a combined horizontal coverage of approximately 165° × 25.1°, resulting in complete horizontal overlap with the camera system and approximately 77\% vertical overlap \cite{thompson2023}.

The decision to employ forward-facing \ac{LiDAR} sensors rather than traditional 360-degree rotating units reflects the research focus on sensor fusion performance.
Unlike mechanically rotating \ac{LiDAR} systems such as the Velodyne HDL-32E, which distribute returns across the full horizontal plane, the Livox Horizon units concentrate their sampling density within the forward-facing camera field of view.
This targeted approach delivers higher effective point cloud density in the region of interest, improving the spatial resolution available for object detection algorithms.
The Livox Horizon scan pattern differs fundamentally from traditional spinning \ac{LiDAR} designs.
Rather than sampling fixed vertical rings that repeat with each rotation, the Livox units employ a rosette-pattern scanning approach that progressively covers the entire field of view over a one-second integration period \cite{thompson2023}.
This non-repetitive scan pattern yields substantially higher spatial sampling density when multiple scans are temporally aggregated, as detailed in Section 3.1.1.3.

All sensor measurements are registered to a common spatial reference frame centered on the vessel platform.
Precise geometric relationships between individual sensors are established through extrinsic calibration procedures detailed in Section 3.2.1.
The \ac{LiDAR} units additionally benefit from factory-configured inter-sensor calibration, allowing all three Livox devices to report points from a unified origin corresponding to the center sensor position.
Position and orientation of the platform itself is provided by a Pinpoint \ac{GPS}/\ac{INS} system, enabling transformation of sensor observations between the vessel reference frame and inertial (world) coordinates.
This capability proves essential for temporal aggregation of \ac{LiDAR} observations during vessel motion, as described in the synchronization methodology of Section 3.2.2.

The perception geometry reflects several key design priorities specific to maritime autonomous navigation research.
Maximum spatial overlap between vision and \ac{LiDAR} modalities enables direct comparison of detection performance and validates late fusion approaches, while the forward-facing emphasis addresses the reality that maritime collision avoidance and navigation planning require detailed perception of the forward corridor, making 360-degree coverage unnecessary for this research application.
Concentrating \ac{LiDAR} sampling within the camera field of view provides higher effective point cloud density compared to omnidirectional sensors with equivalent point return rates.
Furthermore, the Livox scan pattern combined with \ac{GPS}/\ac{INS} integration enables temporal accumulation of observations during vessel motion, further increasing effective spatial resolution.
The resulting sensor configuration provides the geometric foundation for the comparative object detection performance analysis that forms the core of this research.
Detailed specifications of individual sensor components are presented in the following subsections.

            \subsubsection{HDR Camera}
            
            \subsubsection{Livox Horizon}
            
            \subsubsection{Pinpoint GPS/INS}
            
        \subsection{Compute and LAN}
        
    \section{Data Collection}
    
        \subsection{Calibration}
        
            \subsubsection{Camera Intrinsics}
            
            \subsubsection{Camera Extrinsics}
            
            \subsubsection{Livox Intrinsics}
            
        \subsection{Synchronization}
        
            \subsubsection{Clock Synchronization}
            
            \subsubsection{Video Pipeline}
            
            \subsubsection{Temporal Drift}
            
    \section{Data Output}

\chapter{Dataset}

\chapter{Real-time Object Detection}

\chapter{Late Fusion}

\chapter{Conclusions}

% This chapter will synthesize findings from all three research objectives:
% - Summary of comparative performance results between LiDAR and vision systems
% - Calibration and synchronization framework effectiveness
% - Real-time processing capability validation
% - Implications for ASV perception system design
% - Contribution to maritime autonomous systems knowledge

\section{Research Objective Achievement Summary}
% Placeholder for objective completion summary

\section{Performance Comparison Findings}
% Placeholder for key comparative analysis conclusions

\section{Implications for ASV System Design}
% Placeholder for practical design guidance conclusions

\chapter{Recommendations and Future Work}

% This chapter will address:
% - Recommendations for ASV perception system design based on findings
% - Sensor selection guidance for maritime applications
% - Future research directions for maritime sensor fusion
% - Technology transfer opportunities to operational systems

\section{ASV Perception System Design Recommendations}
% Placeholder for design guidance recommendations

\section{Future Research Directions}
% Placeholder for future work recommendations

\section{Technology Transfer Opportunities}
% Placeholder for practical application recommendations




% \printbibliography
\bibliographystyle{plainnat}
% \bibliography{References}
\bibliography{Dissertation}

\backmatter

\chapter{A Test of the Appendix System}

Tables of Results

\chapter{Another Test of the Appendix System}
Supplemental Figures.
\end{document}

